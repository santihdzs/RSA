\section{Introduction}
\label{sec:section1}
In a society where technology is used daily to transmit sensitive information worldwide, data encryption has become increasingly important since it ensures that data transmitted over the internet remains private and secure. Due to this, many cryptographic systems have been implemented to encrypt data over the internet, with one of the oldest and most widely used encryption methods being RSA encryption.

RSA encryption is an algorithm presented in 1977 by Ron Rivest, Adi Shamir, and Leonard Adleman \citep{Rivest_Shamir_Adleman_1977}. RSA’s security relies on the difficulty of factoring the product of two large prime numbers and is an algorithm that implements many mathematical concepts.

This paper mainly falls under two branches of mathematics: number theory, which deals with the study of integers and their properties, and algebraic structures, which involves the study of sets, particularly in the context of integers modulo $n$ ($\mathbb{Z}_n$), where $n$ is a product of two large prime numbers. This paper intends to explore the algorithm behind RSA, along with its mathematical properties, to assess RSA’s security. This paper is structured as follows: In \autoref{sec:section2}, I introduce the underlying mathematics behind the RSA algorithm. In \autoref{sec:section3}, I present the RSA algorithm and explain how and why it works. \autoref{sec:section4} explores a different method for deriving RSA private exponents. \autoref{sec:section5} contains a detailed example of the key generation, encryption, and decryption processes; while in \autoref{sec:section6}, I explain the framework used to assess the security of RSA. Lastly, \autoref{sec:section7} presents concluding remarks and assesses the security of RSA encryption based on the framework presented in \autoref{sec:section6}.




