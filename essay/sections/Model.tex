\section{Underlying Mathematics}\label{sec:section2}
RSA relies on certain mathematical principles that make it work; one of these concepts is the use of prime numbers. RSA encryption uses large primes that are multiplied together to obtain a product. The security of RSA then rests on the fact that it is extremely difficult to factorize this product back into its original prime factors, especially if these are very large. 

Other mathematical concepts in RSA include modular arithmetic, which is used in the encryption and decryption process to perform operations on large primes efficiently, as well as coprime numbers. Because of the underlying mathematics, certain concepts are key to the inner workings of RSA; the mathematical concepts that form the basis of RSA encryption include Fermat's Little Theorem, Euler's Totient Function, and the Carmichael function. 

\subsection{Modular Arithmetic}\label{sec:section2.1}

In mathematics, modular arithmetic is a system of arithmetic for integers where numbers loop when reaching a certain value known as the modulus. An example of modular arithmetic is the 12-hour clock, which divides the day into two 12-hour periods. In 12-hour clocks, if the time is 5:00 now, in 8 hours, the time will be 1:00. Normal addition would result in $5 + 8 = 13$, but 13:00 is expressed as 1:00 on the clock face because clocks “loop” every 12 hours. In this example, it can be said that 13 is congruent to 1 modulo 12, which is written as

$$
13 \equiv 1 \pmod{12}
$$

\subsubsection{Integers Modulo n}\label{sec:section2.1.1}

Integers modulo $n$, denoted $\mathbb{Z}_n$, represent the set of integers from $0$ to $n-1$, where two integers are considered equivalent (or congruent) if their difference is divisible by $n$. In other words, integers $a, b$ are equivalent modulo $n$ (written as $a \equiv b \pmod{n}$) if $a - b$ is a multiple of $n$. Integers modulo $n$ can be expressed using the following notation:

$$
\mathbb{Z}_n = \{ 0,1,2, \ldots, n-1 \}.
$$

For example, $\mathbb{Z}_4 = {0,1,2,3}$ represents the integers modulo 4. Furthermore, $\mathbb{Z}_4$ has the following equivalences:

\[
\begin{aligned}
&1 \equiv 1 \pmod{4} \\
&2 \equiv 2 \pmod{4} \\
&3 \equiv 3 \pmod{4} \\
&4 \equiv 0 \pmod{4}
\end{aligned}
\]

In the example above, $4 \equiv 0 \pmod{4}$ because $4-0 = 4$, which is divisible by $4$. This means that in $\mathbb{Z}_4$, 4 is equivalent to 0 since they differ by a multiple of 4. Similarly, in $\mathbb{Z}_4$, $5 \equiv 1 \pmod{4}, 6 \equiv 2 \pmod{4}$, and so on. Integers modulo $n$ play a key role in RSA key generation.



\subsection{Group Theory}\label{sec:section2.2}

In mathematics, a group is defined as a non-empty set $G$ with a binary operation (denoted "$\cdot$") that combines any two elements $a$ and $b$ of $G$ to form an element of $G$, denoted $a \cdot b$, such that the following three requirements, known as Group axioms, are satisfied. 

\begin{itemize}
    \item \textbf{Associativity} \\
    For all $a,b,c$ in $G$, one has $(a \cdot b) \cdot c = a \cdot (b \cdot c)$ For example, $(1 + 2) + 3 = 1 + (2 + 3)$.
  \end{itemize}

  \begin{itemize}
    \item \textbf{Identity Element} \\
    There exists an element $e$ in $G$ such that, for every $a$ in $G$, one has $e \cdot a = a$ and $a \cdot e = a$, for example, $0 + 1 = 1, 1+ 0 = 1$. (In this example 0 is the group’s identity element).
  \end{itemize}

  \begin{itemize}
    \item \textbf{Inverse Element} \\
    For each $a$ in $G$, there exists an element $b$ in $G$ such that $a \cdot b = e$ and $b \cdot a = e$, where $e$ is the identity element. For each $a$, the element $b$ is unique; this is called the inverse of $a$ and is denoted as $a^{-1}$.  For example, $3 + (-3) = 0, -3 + 3 = 0$, where $a = 3, b=-3, e= 0$
  \end{itemize}
  
  In the group examples above, I described the set of integers under addition, which is commonly denoted as $\mathbb{Z}^{+}$ and satisfies all three of the group axioms. Groups are relevant to RSA encryption, as they are a crucial concept responsible for the relationship between public and private keys and the impossibility of algorithmically deriving a private key from a public key.



  \subsubsection{Cyclic Subgroups}\label{sec:section2.2.1}

  For any element $g$ in any group $G$, one can form the subgroup that consists of all its integer powers, meaning that the subgroup generated by the element $g$ consists of all powers of $g$, where $k$ ranges over all integers, this can be expressed as:
  
  $$
  \langle g \rangle = \{ g^k \mid k \in \mathbb{Z} \}.
  $$
  
  This is known as a cyclic subgroup and is commonly denoted as $\mathbb{Z}_n$. For example, consider a group of integers modulo 4, under addition, denoted as $\mathbb{Z}^{+}_{4}$. This group has four elements: $\{ 0,1,2,3 \}$. If we take the element 1, the powers of 1 in $\mathbb{Z}^{+}_{4}$ are $\{ 1,2,3,0 \}$, where $1^0 = 1, 1^1 = 1, 1^2 = 2, 1^3 = 3, 1^4 = 0$. Since all elements of $\mathbb{Z}^{+}_{4}$ can be generated by powers of 1, we say that $\mathbb{Z}^{+}_{4}$ is a cyclic group generated by 1.
  
  In general, a group is cyclic if an element $g$ exists in the group such that every element of the group can be written as a power of $g$.
  

\subsubsection{Multiplicative Group of Integers Modulo n}\label{sec:section2.2.2}
  
The multiplicative group of integers modulo $n$, denoted $\mathbb{Z}^{*}_{n}$, consists of all integers $a$ that range from 1 to $n-1$ that are coprime to $n$. For example, if $n = 8$, the integers from 1 to 7 are 1,2,3,4,5,6, and 7. However, the numbers that are coprime to 8 in this range are 1,3,5, and 7. Therefore $\mathbb{Z}^{*}_{8} = \{ 1,3,5,7 \}$.
  
\subsubsection{Modular Multiplicative Inverse}\label{sec:section2.2.3}

The modular multiplicative inverse of an integer $a$ modulo $m$ is another integer $b$ such that the product of $a$ and $b$ is congruent to 1 modulo $m$. In other words, $b$ is the multiplicative inverse of $a$ if $a \times b \equiv 1 \pmod{m}$. A modular multiplicative inverse for an integer $a$ modulo $m$ only exists if $a$ and $m$ are coprime, since that would mean that $a$ and $m$ have no common factors other than 1. If $a$ and $m$ have a common factor that is greater than 1, there is no multiplicative inverse since any multiple of $a$ will also be a multiple of the common factor shared with $m$, meaning it cant be congruent to 1. For example, to find the multiplicative inverse of $3 \pmod{7}$, we need to find an integer $b$ such that $3 \times b \equiv 1 \pmod{7}$. If we then try out multiple values for $b$, we find that $3 \times 5 \equiv 15 \equiv 1 \pmod{7}$. Therefore, the modular multiplicative inverse of $3 \pmod{7}$ is 5. The modular multiplicative inverse is a crucial concept in RSA as it ensures that a private key can be used to decrypt a message that has been encrypted using a public key since these are the modular multiplicative inverse of each other.
  

\subsection{Euler's Totient Function}\label{sec:section2.3} 
Euler’s totient function is expressed as $\phi(n)$. It is a function that counts the number of positive integers that are less than or equal to $n$ and coprime to $n$. In other words, it counts the number of totatives of $n$. It can also be expressed as the number of integers $k$ in the range $1 \leq k \leq n, \gcd(n,k)$ \citep{long1972elementary}. Two numbers are said to be coprime if their greatest common divisor $(\gcd)$ is 1. For example, $\phi(9) = 6$, since $1,2,4,5,7,8$, are all integers that are coprime to 9, but 3, 6, and 9 are not. Likewise, $\phi(13) = 12$, as there are 12 numbers that are less than 13 and coprime to 13. 

Euler’s function plays a role in the RSA encryption key generation process. In RSA, an encryption exponent $e$ is chosen such that $e$ is coprime to $\phi(n)$, where $n$ is the product of two prime numbers $(p,q)$. This ensures that a unique decryption exponent $d$ exists, such that $d \times e \equiv 1 (mod\phi(n))$. In other words, the product of the decryption exponent $d$ and the encryption exponent $e$ $(d \times e)$ and 1 both have the same remainder when being divided by $\phi(n)$, which is another way of saying that $d \times e$ is congruent to 1 modulo $\phi(n)$.

\subsection{Carmichael function}\label{sec:section2.4} 
Carmichael’s function is expressed as $\lambda(n)$. It is a function that gives the smallest positive integer $k$ such that $a^{k} \equiv 1 \pmod{n}$ for all integers $a$ that are coprime to $n$ \citep{Carmichael1910}. This means that Charmichael’s function is the exponent of the multiplicative group of integers modulo $n$; in other words, $\lambda(n)$ is the smallest positive integer such that raising any number coprime to $n$ to the power of $\lambda(n)$ results in 1 modulo $n$. For example, $\lambda (5) = 4$, because for any number $0 < a < 5$, that is coprime to 5, in other words, $a \in \{ 1,2,3,4 \},$ the following is true,

\[
\begin{aligned}
    &1^{1} \equiv 1 \pmod{5} \\
    &2^{1} \equiv 2 \pmod{5}, 2^{2} \equiv 4 \pmod{5}, 2^{3} \equiv 3 \pmod{5}, 2^{4} \equiv 1 \pmod{5} \\
    &3^{1} \equiv 3 \pmod{5}, 3^{2} \equiv 4 \pmod{5}, 3^{3} \equiv 2 \pmod{5}, 3^{4} \equiv 1 \pmod{5} \\
    &4^{1} \equiv 4 \pmod{5}, 4^{2} \equiv 1 \pmod{5}, 4^{3} \equiv 4 \pmod{5}, 4^{4} \equiv 1 \pmod{5}
\end{aligned}
\]

In RSA, carmichael’s function is related to Euler’s totient function $(\phi(n))$ and is also used in key generation.

\subsection{Fermat's Little Theorem}\label{sec:section2.5}
Fermat’s Little theorem states that if $p$ is a prime number and $a$ is an integer that is coprime to $p$ then

$$a^{p-1}\equiv 1 \pmod{p}.$$

For example, if $a=2$ and $p=11$, then $2^{10} = 1024-1 = 1023 = 11 \times 93$, meaning 1023 is a multiple of 11. This is a key concept in RSA encryption, as it ensures that the encryption and decryption operations are inverses of each other \citep{fermats}. Furthermore, this mathematical property makes calculating $\lambda(n)$ a straightforward process, since it means that $\lambda(n) = lcm(p-1, q-1)$. This is a crucial part of RSA since $n$ is made public as it is required for both the encryption and decryption process. Since $n$ is public, it is crucial that $\lambda(n)$ cannot be easily calculated, as this would allow an attacker to derive the private key $d$ from $n$. However, since $n$ is a product of two prime numbers, it is practically impossible to calculate $\lambda(n)$ without knowing $p$ and $q$, since it would involve the factorization of $n$.

