\section{RSA Example}\label{sec:section5}

Following the steps outlined in sections \ref{sec:section3} and \ref{sec:section4}, I calculate the following RSA example where $p=61$ and $q=53$. Firstly, I calculate the modulus $n$, where

$$n = 61 \times 53 = 3233$$

The next step is to calculate $\lambda(n)$ which can be done as follows

$$\lambda (3233) = lcm (60,52) = 780$$

This operation holds thanks to the mathematical concepts outlined in section \ref{sec:section2}, particularly Fermat's Little Theorem.
I then choose any number $e$ where $2 < e < 780$ that is coprime to 780. I let $e=17$.
Hence $d=413$, as 

$$
17^{779} \equiv 413 \pmod{780}
$$

Furthermore, $(17 \times 413) \equiv 1 \pmod{780}$. This leaves me with a public key ($n = 3233, e = 17$) and a private exponent ($n = 3233, d = 413$). In order to encrypt $m=2$, for example, I calculate

$$c = 2^{17} \pmod{3233} = 1752$$.

In other words

$$2^{17} \equiv 1752 \pmod{3233}$$

This is calculated using modular exponentiation, which is done by dividing $m^e$ $(2^{17} = 131,072)$ by the modulus $(3233)$.

$$
\frac{131,072}{3233} = 40 + 1752
$$

To decrypt $c = 1752$, I calculate

$$1752^{413} \pmod{3233} = 2$$.

Which can also be expressed as

$$1752^{413} \equiv 2 \pmod{3233}$$

This decryption calculation is also performed using modular exponentiation, however, this was calculated using code since performing these calculations manually is increadibly difficult, not to mention, inefficient.

