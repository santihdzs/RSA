\section{RSA Algorithm}\label{sec:section3}

RSA encryption involves three steps: key generation, encryption, and decryption. RSA relies on the practicality of finding three large positive integers $e$, $d$, and $n$, where $n$ is the product of two large prime numbers. The idea behind RSA is that by using modular exponentiation for any integer $m$ within a range, the following relation holds:

$$(m^e)^d \equiv m \pmod{n}$$

Here, the $\equiv$ symbol denotes modular congruence, meaning that both $(m^e)^d$ and $m$ are congruent $\pmod{n}$. Even when knowing $e$, $n$ or $m$, it can be extremely difficult to find $d$ without knowing $\lambda(n)$. The public key (represented by $n$ and $e$) can be known by anyone and is used to encrypt messages ($m$), while the private key ($d$) can be used to decrypt the message. The following explains all three steps used for RSA, along with an example using small primes.

\subsection{Key Generation}\label{sec:section3.1}

The keys for RSA are generated as follows:

Firstly, two different large prime numbers, $p$ and $q$ are chosen, $n = pq$ is then calculated, $n$ is used as the modulus for both keys and is made public.

Secondly, one calculates $\lambda (n)$, since $n = pq$, and $p$ and $q$ are prime, $\lambda(p)= \phi(p) = p-1$, and likewise $\lambda(q) = q-1$. Hence $\lambda(n) = lcm(p-1, q-1)$. $\lambda(n)$ is kept secret, and the lcm (least common multiple) can be calculated using the Euclidean algorithm since

$$lcm(a,b)= \frac{\left| ab \right|}{gcd \left( a,b \right)}$$.

Afterwards, an integer $e$ is chosen such that $2 < e < \lambda(n)$ and $\gcd(e, \lambda(n))=1$; meaning that $e$ and $\lambda(n)$ are coprime.

\begin{itemize}
  \item $e$ is then released as part of the public key.
\end{itemize}

Finally, $d$ is determined as $d \equiv e^{-1}\pmod{\lambda(n)}$; that is, $d$ is the modular multiplicative inverse of $e$ modulo $\lambda(n)$.

\begin{itemize}
  \item $d$ is kept secret as the \textit{private key exponent}.
\end{itemize}

The public key consists of the modulus $n$ and the encryption exponent $e$, while the private key consists of the decryption exponent $d$, which should be kept secret. $p$, $q$, and $\lambda(n)$ should all be discarded since they can all be used to calculate $d$, and aren't used in the encryption and decryption process.


\subsection{Encryption}\label{sec:section3.2}

To encrypt a text message $M$, $M$ has to be converted into an integer $m$ such that $0 \leq m < n $; this can be done by using an agreed-upon protocol known as a padding scheme, which maps each character to a number. Afterwards, the encrypted message $c$ can be calculated using public key $e$ such that 

$$c \equiv m^e \pmod{n}$$.



\subsection{Decryption}\label{sec:section3.3}

$m$ can then be recovered from $c$, using private key $d$ by calculating

$$c^d \equiv (m^e)^d \equiv m \pmod{n}$$

Given $m$, the original message $M$ can then be recovered by reversing the padding scheme.
