\section{Conclusion}
\label{sec:section7}
This paper studied the behavior of various mathematical concepts within the RSA cryptosystem, and the ways in which they interact with each other to ensure the reliability and security of the encryption process. In addition to this, properties of the cyclic groups that make up RSA were used to uncover possible vulnerabilities, these properties include the relationship between the modulus and the possible encrypted values of a message, as well as the congruence between the public and private exponents within modulo $\lambda (n)$. I placed particular focus on how the observed mathematical relationships could be used to find exploits in the security of the encryption and decryption processes; and the ways in which the same properties can be used to work around these flaws. An interesting approach to further research would be to conduct a similar analysis on other cryptosystems and compare them to RSA.

Overall, this exploration presents meaningful insight that could be beneficial to the security of RSA users, the changes that were proven to strengthen RSA are the use of large prime factors, and the large differences in magnitude between these. 
